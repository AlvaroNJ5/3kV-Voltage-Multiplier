\section{Verificaciones}

En esta sección se va a justificar mediante las simulaciones necesarias que el funcionamiento del circuito es el correcto.
Primero, se comprueba que la tensión de salida sin carga es igual a la esperada. En la figura \ref{Vout_50Hz} se puede ver que, al transcurrir
un tiempo de aproximadamente 5 segundos, la tensión de salida se estabiliza en $3101V$, valor que supone un error mínimo del $0,33\%$
respecto a la tensión de salida esperada sin carga ($3111,27V$), como se puede ver en la siguiente expresión:

\begin{equation}
    E_{V_{noload}} = \frac{3111,27V - 3101V}{3111,27V} \cdot 100 = 0,33\%
\end{equation}

\begin{figure}[H]
    \centering
    \includegraphics[width=1\textwidth]{Imagenes_alvaro/Vout_50hz.png}
    \caption{Tensión de salida y entrada a 50Hz (sin carga)}
    \label{Vout_50hz}
\end{figure}

Además, podemos garantizar que la tensión obtenida está dentro del rango de tensión de salida especificado, puesto que la tensión máxima 
establecida ($5\%$ de $3kV$) es de $3150V$.

Tras comprobar que el circuito funciona correctamente, pasamos a conectar la carga y el circuito de medida.
Respecto del caso anterior, el tiempo de establecimiento es similar y no relevante puesto que no existe ningún requerimiento respecto del mismo, por lo que
se pasa a analizar la tensión de salida en estacionario, obteniendo el resultado \ref{Vout_50hz_load}.

\begin{figure}[H]
    \centering
    \includegraphics[width=1\textwidth]{Imagenes_alvaro/Vout_50hz_load.png}
    \caption{Tensión de salida y entrada a 50Hz con carga}
    \label{Vout_50hz_load}
\end{figure}

Como se puede ver en la figura \ref{Vout_50hz_load}, se obtiene una tensión media de $3044V$, una tensión máxima de $3049V$ y una tensión
mínima de $3039V$. De esta manera, se puede garantizar que el requisito de rizado de $10V$ para 50Hz sí se cumple, y el error obtenido 
para la tensión media es mínimo:

\begin{equation}
    E_{V_{load}} = \frac{3047,93V - 3044V}{3047,93V} \cdot 100 = 0,13\%
\end{equation}

Respecto a la corriente, se analiza en la figura \ref{Corrientes} la corriente que circula a través de la carga y la que circula a través del circuito
de medida. Puesto que, como se explica en la sección anterior, la tensión de salida es ligeramente menor a la esperada, en consecuencia la corriente en la carga también lo será,
midiendo un valor de $4.898mA$. La corriente a través del circuito de medida se ha diseñado para no ser significativa, dando un valor de $21,718\mu A$.
El error cometido en la corriente a través de la carga es el siguiente:

\begin{equation}
    E_{I_{load}} = \frac{5mA - 4,898mA}{5mA} \cdot 100 = 2,04\%
\end{equation}

\begin{figure}[H]
    \centering
    \includegraphics[width=1\textwidth]{Imagenes_alvaro/Corrientes.png}
    \caption{Corriente en la carga y sistema de medida}
    \label{Corrientes}
\end{figure}

Por otra parte, también es necesario analizar la tensión máxima en los semiconductores y condensadores, cuyo resultado se puede ver en la figura \ref{Vmax}.
La tensión máxima es de $618,33V$ (aproximadamente el doble a la de la fuente), mucho menor a los $1000V$ 
establecidos como máximo, por lo que este requerimiento está cumplido. Además, cabe destacar que el primer 
condensador (C1) se carga a una tensión igual a la de la fuente, por lo que se puede utilizar un condensador de menor rango de tensión para este caso.

\begin{figure}[H]
    \centering
    \includegraphics[width=1\textwidth]{Imagenes_alvaro/Vmax.png}
    \caption{Tensión máxima en condensadores y diodos}
    \label{Vmax}
\end{figure}

También es necesario comprobar que la tensión en el multímetro no supera los $200V$ establecidos. Puesto que se definió el divisor para obtener $200V$
para la máxima tensión de salida posible ($3111,27V$), en este caso la tensión medida será necesariamente menor, dando un valor de $195,91V$ como se puede ver en la figura
\ref{V_mult}.

\begin{figure}[H]
    \centering
    \includegraphics[width=1\textwidth]{Imagenes_alvaro/V_mult.png}
    \caption{Tensión en el multímetro}
    \label{V_mult}
\end{figure}

En la figura \ref{200hz} se puede ver la tensión de salida y corriente de carga a 50Hz (gráfica roja), 100Hz (gráfica azul), 150Hz (gráfica verde)
y 200Hz (gráfica azul claro). Como se ha calculado previamente la tensión obtenida aumenta con la frecuencia, lo que produce un aumento de la corriente en consecuencia.
Se obtiene una tensión de salida de $3090V$ y una corriente de carga de $4,963mA$ para una frecuencia de 200Hz. Obtenemos los errores respecto a los valores esperados:

\begin{equation}
    E_{V_{load-200Hz}} = \frac{3095,44V - 3090V}{3095,44V} \cdot 100 = 0,18\%
\end{equation}

\begin{equation}
    E_{I_{load-200Hz}} = \frac{5mA - 4,963mA}{5mA} \cdot 100 = 0,74\%
\end{equation}

\begin{figure}[H]
    \centering
    \includegraphics[width=1\textwidth]{Imagenes_alvaro/200hz.png}
    \caption{Tensión de salida y corriente en la carga para múltiples frecuencias}
    \label{200hz}
\end{figure}

Finalmente, se comprueba al funcionamiento del circuito para una señal de entrada cuadrada de 50Hz, con valores máximos y mínimos iguales a los valores
de pico positivo y negativo de la senoidal y con un ciclo de trabajo del 50\%. El resultado se puede ver en la figura \ref{cuadrada}, donde se obtiene
un valor medio de $3044V$ exactamente igual que el obtenido con la senoidal de misma frecuencia. También se mantiene el mismo rizado de $10V$ del caso anterior.

\begin{figure}[H]
    \centering
    \includegraphics[width=1\textwidth]{Imagenes_alvaro/cuadrada.png}
    \caption{Tensión de salida y entrada con señal de entrada de onda cuadrada}
    \label{cuadrada}
\end{figure}